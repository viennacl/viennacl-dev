
\chapter{Configuring OpenCL Contexts and Devices} \label{chap:multi-devices}
Support for multiple devices was officially added in {\OpenCL} 1.1.
Among other things, this allows e.g.~to use all CPUs in a multi-socket CPU mainboard as a single {\OpenCL} compute device.
Nevertheless, the efficient use of multiple {\OpenCL} devices is far from trivial, because algorithms have to be designed such that
they take distributed memory and synchronization issues into account.

Support for multiple {\OpenCL} devices and contexts was introduced in {\ViennaCL} with version 1.1.0. In the following we give a description of the
provided functionality.

\NOTE{In {\ViennaCLversion} there is no native support for automatically executing operations over multiple GPUs. Partition of data is left to the user.}

\section{Context Setup}
Unless specified otherwise (see Chap.~\ref{chap:custom-contexts}), {\ViennaCL} silently creates its own context and adds all available default devices with a single queue per device to it.
All operations are then carried out on this context, which can be obtained with the call
\begin{lstlisting}
 viennacl::ocl::current_context();
\end{lstlisting}
This default context is identified by the ID $0$ (of type \lstinline|long|).
{\ViennaCL} uses the first platform returned by the OpenCL backend for the context.
If a different platform should be used on a machine with multiple platforms available,
this can be achieved with
\begin{lstlisting}
 viennacl::ocl::set_context_platform_index(id, platform_index);
\end{lstlisting}
where the context ID is \lstinline|id| and \lstinline|platform_index| refers to the array index of the platform as returned by \lstinline|clGetPlatformIDs()|.

By default, only the first device in the context is used for all operations. This device can be obtained via
\begin{lstlisting}
 viennacl::ocl::current_context().current_device();
 viennacl::ocl::current_device(); //equivalent to above
\end{lstlisting}
A user may wish to use multiple {\OpenCL} contexts, where each context consists of a subset of the available devices.
To setup a context with ID \lstinline|id| with a particular device type only, the user has to specify this
prior to any other {\ViennaCL} related statements:
\begin{lstlisting}
//use only GPUs:
viennacl::ocl::set_context_device_type(id, viennacl::ocl::gpu_tag());
//use only CPUs:
viennacl::ocl::set_context_device_type(id, viennacl::ocl::cpu_tag());
//use only the default device type
viennacl::ocl::set_context_device_type(id, viennacl::ocl::default_tag());
//use only accelerators:
viennacl::ocl::set_context_device_type(id, viennacl::ocl::accelerator_tag());
\end{lstlisting}
Instead of using the tag classes, the respective {\OpenCL} constants \texttt{CL\_DEVICE\_TYPE\_GPU} etc.~can be supplied as second argument.

Another possibility is to query all devices from the current platform:
\begin{lstlisting}
std::vector< viennacl::ocl::device > devices =
    viennacl::ocl::platform().devices();
\end{lstlisting}
and create a custom subset of devices, which is then passed to the context setup routine:
\begin{lstlisting}
//take the first and the third available device from 'devices'
std::vector< viennacl::ocl::device > my_devices;
my_devices.push_back(devices[0]);
my_devices.push_back(devices[2]);

//Initialize the context with ID 'id' with these devices:
viennacl::ocl::setup_context(id, my_devices);
\end{lstlisting}
Similarly, contexts with other IDs can be set up.

\TIP{For details on how to initialize {\ViennaCL} with already existing contexts, see Chapter \ref{chap:custom-contexts}.}

The library user is reminded that memory objects within a context are allocated for all devices within a context.
Thus, setting up contexts with one device each is optimal in terms of memory usage, because each memory object is then bound to a single device only.
However, memory transfer between contexts (and thus devices) has to be done manually by the library user then. Moreover, the user has
to keep track in which context the individual {\ViennaCL} objects have been created, because all operands are assumed to be in the currently active context.

\section{Switching Contexts and Devices}
{\ViennaCL} always uses the currently active {\OpenCL} context with the currently active device to enqueue compute kernels. The default context is identified by ID '$0$'.
The context with ID \lstinline|id| can be set as active context with the line.
\begin{lstlisting}
viennacl::ocl::switch_context(id);
\end{lstlisting}
Subsequent kernels are then enqueued on the active device for that particular context.

Similar to setting contexts active, the active device can be set for each context. For example, setting the second device in the context to be the active device, the lines
\begin{lstlisting}
viennacl::ocl::current_context().switch_device(1);
\end{lstlisting}
are required. In some circumstances one may want to pass the device object directly, e.g.~to set the second device of the platform active:
\begin{lstlisting}
std::vector<viennacl::ocl::device> const & devices =
    viennacl::ocl::platform().devices();
viennacl::ocl::current_context().switch_device(devices[1]);
\end{lstlisting}
If the supplied device is not part of the context, an error message is printed and the active device remains unchanged.


\section{Setting OpenCL Compiler Flags}
Each {\OpenCL} context provides a member function \lstinline|.build_options()|, which can be used to pass OpenCL compiler flags prior to compilation.
Note that flags need to be passed to the context prior to the compilation of the respective kernels, i.e.~prior the first instantiation of the respective matrix or vector types.

To pass the \lstinline|-cl-mad-enable| flag to the current context, the line
\begin{lstlisting}
 viennacl::ocl::current_context().build_options("-cl-mad-enable");
\end{lstlisting}
is sufficient. Confer to the {\OpenCL} standard for a full list of flags.
