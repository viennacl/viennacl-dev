
\chapter{Change Logs} %\addcontentsline{toc}{chapter}{Change Logs}

\section*{Version 1.4.x}

\subsection*{Version 1.4.2}
This is a maintenance release, particularly resolving compilation problems with Visual Studio 2012.
\begin{itemize}
 \item Largely refactored the internal code base, unifying code for \lstinline|vector|, \lstinline|vector_range|, and \lstinline|vector_slice|.
       Similar code refactoring was applied to \lstinline|matrix|, \lstinline|matrix_range|, and \lstinline|matrix_slice|.
       This not only resolves the problems in VS 2012, but also leads to shorter compilation times and a smaller code base.
 \item Improved performance of matrix-vector products of \lstinline|compressed_matrix| on CPUs using OpenCL.
 \item Resolved a bug which shows up if certain rows and columns of a \lstinline|compressed_matrix| are empty and the matrix is copied back to host.
 \item Fixed a bug and improved performance of GMRES. Thanks to Ivan Komarov for reporting via sourceforge.
 \item Added additional Doxygen documentation.
\end{itemize}


\subsection*{Version 1.4.1}
This release focuses on improved stability and performance on AMD devices rather than introducing new features:
\begin{itemize}
 \item Included fast matrix-matrix multiplication kernel for AMD's Tahiti GPUs if matrix dimensions are a multiple of 128.
       Our sample HD7970 reaches over 1.3 TFLOPs in single precision and 200 GFLOPs in double precision (counting multiplications and additions as separate operations).
 \item All benchmark FLOPs are now using the common convention of counting multiplications and additions separately (ignoring fused multiply-add).
 \item Fixed a bug for matrix-matrix multiplication with \lstinline|matrix_slice<>| when slice dimensions are multiples of 64.
 \item Improved detection logic for Intel OpenCL SDK.
 \item Fixed issues when resizing an empty \lstinline|compressed_matrix|.
 \item Fixes and improved support for BLAS-1-type operations on dense matrices and vectors.
 \item Vector expressions can now be passed to \lstinline|inner_prod()| and \lstinline|norm_1()|, \lstinline|norm_2()| and \lstinline|norm_inf()| directly.
 \item Improved performance when using OpenMP.
 \item Better support for Intel Xeon Phi (MIC).
 \item Resolved problems when using OpenCL for CPUs if the number of cores is not a power of 2.
 \item Fixed a flaw when using AMG in debug mode. Thanks to Jakub Pola for reporting.
 \item Removed accidental external linkage (invalidating header-only model) of SPAI-related functions. Thanks again to Jakub Pola.
 \item Fixed issues with copy back to host when OpenCL handles are passed to CTORs of vector, matrix, or \lstinline|compressed_matrix|. Thanks again to Jakub Pola.
 \item Added fix for segfaults on program exit when providing custom OpenCL queues. Thanks to Denis Demidov for reporting.
 \item Fixed bug in \lstinline|copy()| to \lstinline|hyb_matrix| as reported by Denis Demidov (thanks!).
 \item Added an overload for \lstinline|result_of::alignment| for \lstinline|vector_expression|. Thanks again to Denis Demidov.
 \item Added SSE-enabled code contributed by Alex Christensen.
\end{itemize}



\subsection*{Version 1.4.0}
The transition from 1.3.x to 1.4.x features the largest number of additions, improvements, and cleanups since the initial release.
In particular, host-, OpenCL-, and CUDA-based execution is now supported. OpenCL now needs to be enabled explicitly!
New features and feature improvements are as follows:
\begin{itemize}
 \item Added host-based and CUDA-enabled operations on ViennaCL objects. The default is now a host-based execution for reasons of compatibility.
       Enable OpenCL- or CUDA-based execution by defining the preprocessor constant \lstinline|VIENNACL_WITH_OPENCL| and \lstinline|VIENNACL_WITH_CUDA| respectively.
       Note that CUDA-based execution requires the use of nvcc.
 \item Added mixed-precision CG solver (OpenCL-based).
 \item Greatly improved performance of ILU0 and ILUT preconditioners (up to 10-fold). Also fixed a bug in ILUT.
 \item Added initializer types from Boost.uBLAS (\lstinline|unit_vector|, \lstinline|zero_vector|, \lstinline|scalar_vector|, \lstinline|identity_matrix|, \lstinline|zero_matrix|, \lstinline|scalar_matrix|).
       Thanks to Karsten Ahnert for suggesting the feature.
 \item Added incomplete Cholesky factorization preconditioner.
 \item Added element-wise operations for vectors as available in Boost.uBLAS (\lstinline|element_prod|, \lstinline|element_div|).
 \item Added restart-after-N-cycles option to BiCGStab.
 \item Added level-scheduling for ILU-preconditioners. Performance strongly depends on matrix pattern.
 \item Added least-squares example including a function \lstinline|inplace_qr_apply_trans_Q()| to compute the right hand side vector $Q^T b$ without rebuilding $Q$.
 \item Improved performance of LU-factorization of dense matrices.
 \item Improved dense matrix-vector multiplication performance (thanks to Philippe Tillet).
 \item Reduced overhead when copying to/from \lstinline|ublas::compressed_matrix|.
 \item ViennaCL objects (scalar, vector, etc.) can now be used as global variables (thanks to an anonymous user on the support-mailinglist).
 \item Refurbished OpenCL vector kernels backend.
       All operations of the type v1 = a v2 @ b v3 with vectors v1, v2, v3 and scalars a and b including += and -= instead of = are now temporary-free. Similarly for matrices.
 \item \lstinline|matrix_range| and \lstinline|matrix_slice| as well as \lstinline|vector_range| and \lstinline|vector_slice| can now be used and mixed completely seamlessly with all standard operations except \lstinline|lu_factorize()|.
 \item Fixed a bug when using copy() with iterators on vector proxy objects.
 \item Final reduction step in \lstinline|inner_prod()| and norms is now computed on CPU if the result is a CPU scalar.
 \item Reduced kernel launch overhead of simple vector kernels by packing multiple kernel arguments together.
 \item Updated SVD code and added routines for the computation of symmetric eigenvalues using OpenCL.
 \item \lstinline|custom_operation|'s constructor now support multiple arguments, allowing multiple expression to be packed in the same kernel for improved performances. However, all the datastructures in the multiple operations must have the same size.
 \item Further improvements to the OpenCL kernel generator: Added a repeat feature for generating loops inside a kernel,
       added element-wise products and division, added support for every one-argument OpenCL function.
 \item The name of the operation is now a mandatory argument of the constructor of \lstinline|custom_operation|.
 \item Improved performances of the generated matrix-vector product code.
 \item Updated interfacing code for the Eigen library, now working with Eigen 3.x.y.
 \item Converter in source-release now depends on Boost.filesystem3 instead of Boost.filesystem2, thus requiring Boost 1.44 or above.
\end{itemize}





\section*{Version 1.3.x}

\subsection*{Version 1.3.1}
The following bugfixes and enhancements have been applied:
\begin{itemize}
 \item Fixed a compilation problem with GCC 4.7 caused by the wrong order of function declarations. Also removed unnecessary indirections and unused variables.
 \item Improved out-of-source build in the src-version (for packagers).
 \item Added virtual destructor in the \lstinline|runtime_wrapper|-class in the kernel generator.
 \item Extended flexibility of submatrix and subvector proxies (ranges, slices).
 \item Block-ILU for \lstinline|compressed_matrix| is now applied on the GPU during the solver cycle phase. However, for the moment the implementation file in \newline \texttt{viennacl/linalg/detail/ilu/opencl\_block\_ilu.hpp} needs to be included separately in order to avoid an OpenCL dependency for all ILU implementations.
 \item SVD now supports double precision.
 \item Slighly adjusted the interface for NMF. The approximation rank is now specified by the supplied matrices $W$ and $H$.
 \item Fixed a problem with matrix-matrix products if the result matrix is not initialized properly (thanks to Laszlo Marak for finding the issue and a fix).
 \item The operations $C += prod(A, B)$ and $C -= prod(A, B)$ for matrices A, B, and C no longer introduce temporaries if the three matrices are distinct.
\end{itemize}



\subsection*{Version 1.3.0}
Several new features enter this new minor version release.
Some of the experimental features introduced in 1.2.0 keep their experimental state in 1.3.x due to the short time since 1.2.0, with exceptions listed below along with the new features:
\begin{itemize}
 \item Full support for ranges and slices for dense matrices and vectors (no longer experimental)
 \item QR factorization now possible for arbitrary matrix sizes (no longer experimental)
 \item Further improved matrix-matrix multiplication performance for matrix dimensions which are a multiple of 64 (particularly improves performance for NVIDIA GPUs)
 \item Added Lanczos and power iteration method for eigenvalue computations of dense and sparse matrices (experimental, contributed by G\"unther Mader and Astrid Rupp)
 \item Added singular value decomposition in single precision (experimental, contributed by Volodymyr Kysenko)
 \item Two new ILU-preconditioners added: ILU0 (contributed by Evan Bollig) and a block-diagonal ILU preconditioner using either ILUT or ILU0 for each block. Both preconditioners are computed entirely on the CPU.
 \item Automated OpenCL kernel generator based on high-level operation specifications added (many thanks to Philippe Tillet who had a lot of \emph{fun fun fun} working on this)
 \item Two new sparse matrix types (by Volodymyr Kysenko): \lstinline|ell_matrix| for the ELL format and \lstinline|hyb_matrix| for a hybrid format (contributed by Volodymyr Kysenko).
 \item Added possibility to specify the OpenCL platform used by a context
 \item Build options for the OpenCL compiler can now be supplied to a context (thanks to Krzysztof Bzowski for the suggestion)
 \item Added nonnegative matrix factorization by Lee and Seoung (contributed by Volodymyr Kysenko).
\end{itemize}



\section*{Version 1.2.x}

\subsection*{Version 1.2.1}
The current release mostly provides a few bug fixes for experimental features introduced in 1.2.0.
In addition, performance improvements for matrix-matrix multiplications are applied.
The main changes (in addition to some internal adjustments) are as follows:
\begin{itemize}
 \item Fixed problems with double precision on AMD GPUs supporting \lstinline|cl_amd_fp64| instead of \lstinline|cl_khr_fp64| (thanks to Sylvain R.)
 \item Considerable improvements in the handling of \lstinline|matrix_range|. Added project() function for convenience (cf. Boost.uBLAS)
 \item Further improvements of matrix-matrix multiplication performance (contributed by Volodymyr Kysenko)
 \item Improved performance of QR factorization
 \item Added direct element access to elements of \lstinline|compressed_matrix| using \lstinline|operator()| (thanks to sourceforge.net user Sulif for the hint)
 \item Fixed incorrect matrix dimensions obtained with the transfer of non-square sparse Eigen and MTL matrices to ViennaCL objects (thanks to sourceforge.net user ggrocca for pointing at this)
\end{itemize}





\subsection*{Version 1.2.0}
Many new features from the Google Summer of Code and the I$\mu$E Summer of Code enter this release.
Due to their complexity, they are for the moment still in \textit{experimental} state (see the respective chapters for details) and are expected to reach maturity with the 1.3.0 release.
Shorter release cycles are planned for the near future.
\begin{itemize}
 \item Added a bunch of algebraic multigrid preconditioner variants (contributed by Markus Wagner)
 \item Added (factored) sparse approximate inverse preconditioner (SPAI, contributed by Nikolay Lukash)
 \item Added fast Fourier transform (FFT) for vector sizes with a power of two, standard Fourier transform for other sizes (contributed by Volodymyr Kysenko)
 \item Additional structured matrix classes for circulant matrices, Hankel matrices, Toeplitz matrices and Vandermonde matrices (contributed by Volodymyr Kysenko)
 \item Added reordering algorithms (Cuthill-McKee and Gibbs-Poole-Stockmeyer, contributed by Philipp Grabenweger)
 \item Refurbished CMake build system (thanks to Michael Wild)
 \item Added matrix and vector proxy objects for submatrix and subvector manipulation
 \item Added (possibly GPU-assisted) QR factorization
 \item Per default, a \lstinline|viennacl::ocl::context| now consists of one device only. The rationale is to provide better out-of-the-box support for machines with hybrid graphics (two GPUs), where one GPU may not be capable of double precision support.
 \item Fixed problems with \lstinline|viennacl::compressed_matrix| which occurred if the number of rows and columns differed
 \item Improved documentation for the case of multiple custom kernels within a program
 \item Improved matrix-matrix multiplication kernels (may lead to up to 20 percent performance gains)
 \item Fixed problems in GMRES for small matrices (dimensions smaller than the maximum number of Krylov vectors)
\end{itemize}



\section*{Version 1.1.x}

\subsection*{Version 1.1.2}
This final release of the {\ViennaCL} 1.1.x family focuses on refurbishing existing functionality:
\begin{itemize}
 \item Fixed a bug with partial vector copies from CPU to GPU (thanks to sourceforge.net user kaiwen).
 \item Corrected error estimations in CG and BiCGStab iterative solvers (thanks to Riccardo Rossi for the hint).
 \item Improved performance of CG and BiCGStab as well as Jacobi and row-scaling preconditioners considerably (thanks to Farshid Mossaiby and Riccardo Rossi for a lot of input).
 \item Corrected linker statements in CMakeLists.txt for MacOS (thanks to Eric Christiansen).
 \item Improved handling of {\ViennaCL} types (direct construction, output streaming of matrix- and vector-expressions, etc.).
 \item Updated old code in the \texttt{coordinate\_matrix} type and improved performance (thanks to Dongdong Li for finding this).
 \item Using \lstinline|size_t| instead of \lstinline|unsigned int| for the size type on the host.
 \item Updated double precision support detection for AMD hardware.
 \item Fixed a name clash in direct\_solve.hpp and ilu.hpp (thanks to sourceforge.net user random).
 \item Prevented unsupported assignments and copies of sparse matrix types (thanks to sourceforge.net user kszyh).
\end{itemize}

\subsection*{Version 1.1.1}
This new revision release has a focus on better interaction with other linear algebra libraries. The few known glitches with version 1.1.0 are now removed.
\begin{itemize}
 \item Fixed compilation problems on MacOS X and {\OpenCL} 1.0 header files due to undefined an preprocessor constant (thanks to Vlad-Andrei Lazar and Evan Bollig for reporting this)
 \item Removed the accidental external linkage for three functions (we appreciate the report by Gordon Stevenson).
 \item New out-of-the-box support for {\Eigen} \cite{eigen} and {\MTL} \cite{mtl4} libraries. Iterative solvers from ViennaCL can now directly be used with both libraries.
 \item Fixed a problem with GMRES when system matrix is smaller than the maximum Krylov space dimension.
 \item Better default parameter for BLAS3 routines leads to higher performance for matrix-matrix-products.
 \item Added benchmark for dense matrix-matrix products (BLAS3 routines).
 \item Added viennacl-info example that displays infos about the {\OpenCL} backend used by {\ViennaCL}.
 \item Cleaned up CMakeLists.txt in order to selectively enable builds that rely on external libraries.
 \item More than one installed {\OpenCL} platform is now allowed (thanks to Aditya Patel).
\end{itemize}


\subsection*{Version 1.1.0}
A large number of new features and improvements over the 1.0.5 release are now available:
\begin{itemize}
 \item The completely rewritten {\OpenCL} back-end allows for multiple contexts, multiple devices and even to wrap existing OpenCL resources into ViennaCL objects. A tutorial demonstrates the new functionality. Thanks to Josip Basic for pushing us into that direction.
 \item The tutorials are now named according to their purpose.
 \item The dense matrix type now supports both row-major and column-major
storage.
 \item Dense and sparse matrix types now now be filled using STL-emulated types (\lstinline|std::vector< std::vector<NumericT> >| and \lstinline|std::vector< std::map< unsigned int, NumericT> >|)
 \item BLAS level 3 functionality is now complete. We are very happy with the general out-of-the-box performance of matrix-matrix-products, even though it cannot beat the extremely tuned implementations tailored to certain matrix sizes on a particular device yet.
 \item An automated performance tuning environment allows an optimization of the kernel parameters for the library user's machine. Best parameters can be obtained from a tuning run and stored in a XML file and read at program startup using pugixml.
 \item Two new preconditioners are now included: A Jacobi preconditioner and a row-scaling preconditioner. In contrast to ILUT, they are applied on the OpenCL device directly.
 \item Clean compilation of all examples under Visual Studio 2005 (we recommend newer compilers though...).
 \item Error handling is now carried out using C++ exceptions.
 \item Matrix Market now uses index base 1 per default (thanks to Evan Bollig for reporting that)
 \item Improved performance of norm\_X kernels.
 \item Iterative solver tags now have consistent constructors: First argument is the relative tolerance, second argument is the maximum number of total iterations. Other arguments depend on the respective solver.
 \item A few minor improvements here and there (thanks go to Riccardo Rossi and anonymous sourceforge.net users for reporting the issues)
\end{itemize}

\section*{Version 1.0.x}

\subsection*{Version 1.0.5}
This is the last 1.0.x release. The main changes are as follows:
\begin{itemize}
 \item Added a reader and writer for MatrixMarket files (thanks to Evan Bollig for suggesting that)
 \item Eliminated a bug that caused the upper triangular direct solver to fail on NVIDIA hardware for large matrices (thanks to Andrew Melfi for finding that)
 \item The number of iterations and the final estimated error can now be obtained from iterative solver tags.
 \item Improvements provided by Klaus Schnass are included in the developer converter script (OpenCL kernels to C++ header)
 \item Disabled the use of reference counting for OpenCL handles on Mac OS X (caused seg faults on program exit)
\end{itemize}

\subsection*{Version 1.0.4}
The changes in this release are:
\begin{itemize}
 \item All tutorials now work out-of the box with Visual Studio 2008.
 \item Eliminated all {\ViennaCL} related warnings when compiling with Visual Studio 2008.
 \item Better (experimental) support for double precision on ATI GPUs, but no \texttt{norm\_1}, \texttt{norm\_2}, \texttt{norm\_inf} and \texttt{index\_norm\_inf} functions using ATI Stream SDK on GPUs in double precision.
 \item Fixed a bug in GMRES that caused segmentation faults under Windows.
 \item Fixed a bug in \texttt{const\_sparse\_matrix\_adapter} (thanks to Abhinav Golas and Nico Galoppo for almost simultaneous emails on that)
 \item Corrected incorrect return values in the sparse matrix regression test suite (thanks to Klaus Schnass for the hint)
\end{itemize}

\subsection*{Version 1.0.3}
The main improvements in this release are:
\begin{itemize}
 \item Support for multi-core CPUs with ATI Stream SDK (thanks to Riccardo Rossi, UPC. BARCELONA TECH, for suggesting this)
 \item \lstinline|inner_prod| is now up to a factor of four faster (thanks to Serban Georgescu, ETH, for pointing the poor performance of the old implementation out)
 \item Fixed a bug with \lstinline|plane_rotation| that caused system freezes with ATI GPUs.
 \item Extended the doxygen generated reference documentation
\end{itemize}

\subsection*{Version 1.0.2}
A bug-fix release that resolves some problems with the Visual C++ compiler.
\begin{itemize}
 \item Fixed some compilation problems under Visual C++ (version 2005 and 2008).
 \item All tutorials accidentally relied on {\ublas}. Now \texttt{tut1} and \texttt{tut5} can be compiled without {\ublas}
 \item Renamed \texttt{aux/} folder to \texttt{auxiliary/} (caused some problems on windows machines)
\end{itemize}

\subsection*{Version 1.0.1}
This is a quite large revision of \texttt{ViennaCL 1.0.0}, but mainly improves things under the hood.
\begin{itemize}
 \item Fixed a bug in lu\_substitute for dense matrices
 \item Changed iterative solver behavior to stop if a certain relative residual is reached
 \item ILU preconditioning is now fully done on the CPU, because this gives best overall performance
 \item All OpenCL handles of {\ViennaCL} types can now be accessed via member function \texttt{handle()}
 \item Improved GPU performance of GMRES by about a factor of two.
 \item Added generic \texttt{norm\_2} function in header file \texttt{norm\_2.hpp}
 \item Wrapper for \texttt{clFlush()} and \texttt{clFinish()} added
 \item Device information can be queried by \texttt{device.info()}
 \item Extended documentation and tutorials
\end{itemize}

\subsection*{Version 1.0.0}
First release
